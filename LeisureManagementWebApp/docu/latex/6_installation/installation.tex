\section{Installation}

\begin{enumerate}
	\item Installation von MySQL: \url{https://dev.mysql.com/downloads/installer/}. Einfach dem Installations-Wizard folgen, und das Paket mit den empfohlenen Einstellungen installieren. Für den Admin muss der Name \verb|root|, und als Passwort auch \verb|root| verwendet werden. Danach muss über den MySQL Command Line Client die Datenbank \verb|springbootdb| angelegt werden. Das funktioniert über den Befehl \verb|create database test;|. Alternativ können Admin, Passwort und Datenbankname in der Datei 
	\begin{lstlisting}
		...\LeisureManagementWebApp\backend\src\main\java\backend_main\configuration\PersistenceConfiguration.java
	\end{lstlisting}
	geändert werden.
	\item Ausführen des Skripts \verb|run_webapp.bat|.
	\item Öffnen der Datei \verb|\LeisureManagementWebApp\frontend\index.html| in einem Browser.
	\item Spaß Haben, abbrechen mit \verb|strg + c|.
\end{enumerate}